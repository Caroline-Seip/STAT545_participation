\documentclass[]{article}
\usepackage{lmodern}
\usepackage{amssymb,amsmath}
\usepackage{ifxetex,ifluatex}
\usepackage{fixltx2e} % provides \textsubscript
\ifnum 0\ifxetex 1\fi\ifluatex 1\fi=0 % if pdftex
  \usepackage[T1]{fontenc}
  \usepackage[utf8]{inputenc}
\else % if luatex or xelatex
  \ifxetex
    \usepackage{mathspec}
  \else
    \usepackage{fontspec}
  \fi
  \defaultfontfeatures{Ligatures=TeX,Scale=MatchLowercase}
\fi
% use upquote if available, for straight quotes in verbatim environments
\IfFileExists{upquote.sty}{\usepackage{upquote}}{}
% use microtype if available
\IfFileExists{microtype.sty}{%
\usepackage{microtype}
\UseMicrotypeSet[protrusion]{basicmath} % disable protrusion for tt fonts
}{}
\usepackage[margin=1in]{geometry}
\usepackage{hyperref}
\hypersetup{unicode=true,
            pdftitle={cm011 Exercises: R as a programming language},
            pdfborder={0 0 0},
            breaklinks=true}
\urlstyle{same}  % don't use monospace font for urls
\usepackage{color}
\usepackage{fancyvrb}
\newcommand{\VerbBar}{|}
\newcommand{\VERB}{\Verb[commandchars=\\\{\}]}
\DefineVerbatimEnvironment{Highlighting}{Verbatim}{commandchars=\\\{\}}
% Add ',fontsize=\small' for more characters per line
\usepackage{framed}
\definecolor{shadecolor}{RGB}{248,248,248}
\newenvironment{Shaded}{\begin{snugshade}}{\end{snugshade}}
\newcommand{\KeywordTok}[1]{\textcolor[rgb]{0.13,0.29,0.53}{\textbf{#1}}}
\newcommand{\DataTypeTok}[1]{\textcolor[rgb]{0.13,0.29,0.53}{#1}}
\newcommand{\DecValTok}[1]{\textcolor[rgb]{0.00,0.00,0.81}{#1}}
\newcommand{\BaseNTok}[1]{\textcolor[rgb]{0.00,0.00,0.81}{#1}}
\newcommand{\FloatTok}[1]{\textcolor[rgb]{0.00,0.00,0.81}{#1}}
\newcommand{\ConstantTok}[1]{\textcolor[rgb]{0.00,0.00,0.00}{#1}}
\newcommand{\CharTok}[1]{\textcolor[rgb]{0.31,0.60,0.02}{#1}}
\newcommand{\SpecialCharTok}[1]{\textcolor[rgb]{0.00,0.00,0.00}{#1}}
\newcommand{\StringTok}[1]{\textcolor[rgb]{0.31,0.60,0.02}{#1}}
\newcommand{\VerbatimStringTok}[1]{\textcolor[rgb]{0.31,0.60,0.02}{#1}}
\newcommand{\SpecialStringTok}[1]{\textcolor[rgb]{0.31,0.60,0.02}{#1}}
\newcommand{\ImportTok}[1]{#1}
\newcommand{\CommentTok}[1]{\textcolor[rgb]{0.56,0.35,0.01}{\textit{#1}}}
\newcommand{\DocumentationTok}[1]{\textcolor[rgb]{0.56,0.35,0.01}{\textbf{\textit{#1}}}}
\newcommand{\AnnotationTok}[1]{\textcolor[rgb]{0.56,0.35,0.01}{\textbf{\textit{#1}}}}
\newcommand{\CommentVarTok}[1]{\textcolor[rgb]{0.56,0.35,0.01}{\textbf{\textit{#1}}}}
\newcommand{\OtherTok}[1]{\textcolor[rgb]{0.56,0.35,0.01}{#1}}
\newcommand{\FunctionTok}[1]{\textcolor[rgb]{0.00,0.00,0.00}{#1}}
\newcommand{\VariableTok}[1]{\textcolor[rgb]{0.00,0.00,0.00}{#1}}
\newcommand{\ControlFlowTok}[1]{\textcolor[rgb]{0.13,0.29,0.53}{\textbf{#1}}}
\newcommand{\OperatorTok}[1]{\textcolor[rgb]{0.81,0.36,0.00}{\textbf{#1}}}
\newcommand{\BuiltInTok}[1]{#1}
\newcommand{\ExtensionTok}[1]{#1}
\newcommand{\PreprocessorTok}[1]{\textcolor[rgb]{0.56,0.35,0.01}{\textit{#1}}}
\newcommand{\AttributeTok}[1]{\textcolor[rgb]{0.77,0.63,0.00}{#1}}
\newcommand{\RegionMarkerTok}[1]{#1}
\newcommand{\InformationTok}[1]{\textcolor[rgb]{0.56,0.35,0.01}{\textbf{\textit{#1}}}}
\newcommand{\WarningTok}[1]{\textcolor[rgb]{0.56,0.35,0.01}{\textbf{\textit{#1}}}}
\newcommand{\AlertTok}[1]{\textcolor[rgb]{0.94,0.16,0.16}{#1}}
\newcommand{\ErrorTok}[1]{\textcolor[rgb]{0.64,0.00,0.00}{\textbf{#1}}}
\newcommand{\NormalTok}[1]{#1}
\usepackage{graphicx,grffile}
\makeatletter
\def\maxwidth{\ifdim\Gin@nat@width>\linewidth\linewidth\else\Gin@nat@width\fi}
\def\maxheight{\ifdim\Gin@nat@height>\textheight\textheight\else\Gin@nat@height\fi}
\makeatother
% Scale images if necessary, so that they will not overflow the page
% margins by default, and it is still possible to overwrite the defaults
% using explicit options in \includegraphics[width, height, ...]{}
\setkeys{Gin}{width=\maxwidth,height=\maxheight,keepaspectratio}
\IfFileExists{parskip.sty}{%
\usepackage{parskip}
}{% else
\setlength{\parindent}{0pt}
\setlength{\parskip}{6pt plus 2pt minus 1pt}
}
\setlength{\emergencystretch}{3em}  % prevent overfull lines
\providecommand{\tightlist}{%
  \setlength{\itemsep}{0pt}\setlength{\parskip}{0pt}}
\setcounter{secnumdepth}{0}
% Redefines (sub)paragraphs to behave more like sections
\ifx\paragraph\undefined\else
\let\oldparagraph\paragraph
\renewcommand{\paragraph}[1]{\oldparagraph{#1}\mbox{}}
\fi
\ifx\subparagraph\undefined\else
\let\oldsubparagraph\subparagraph
\renewcommand{\subparagraph}[1]{\oldsubparagraph{#1}\mbox{}}
\fi

%%% Use protect on footnotes to avoid problems with footnotes in titles
\let\rmarkdownfootnote\footnote%
\def\footnote{\protect\rmarkdownfootnote}

%%% Change title format to be more compact
\usepackage{titling}

% Create subtitle command for use in maketitle
\newcommand{\subtitle}[1]{
  \posttitle{
    \begin{center}\large#1\end{center}
    }
}

\setlength{\droptitle}{-2em}

  \title{cm011 Exercises: R as a programming language}
    \pretitle{\vspace{\droptitle}\centering\huge}
  \posttitle{\par}
    \author{}
    \preauthor{}\postauthor{}
    \date{}
    \predate{}\postdate{}
  

\begin{document}
\maketitle

\section{Part I}\label{part-i}

\subsection{Types and Modes and Classes, Oh
My!}\label{types-and-modes-and-classes-oh-my}

R objects have a \emph{type}, a \emph{mode}, and a \emph{class}. This
can be confusing:

\begin{Shaded}
\begin{Highlighting}[]
\NormalTok{a <-}\StringTok{ }\DecValTok{3}
\KeywordTok{print}\NormalTok{(}\KeywordTok{typeof}\NormalTok{(a))}
\end{Highlighting}
\end{Shaded}

\begin{verbatim}
## [1] "double"
\end{verbatim}

\begin{Shaded}
\begin{Highlighting}[]
\KeywordTok{print}\NormalTok{(}\KeywordTok{mode}\NormalTok{(a))}
\end{Highlighting}
\end{Shaded}

\begin{verbatim}
## [1] "numeric"
\end{verbatim}

\begin{Shaded}
\begin{Highlighting}[]
\KeywordTok{print}\NormalTok{(}\KeywordTok{class}\NormalTok{(a))}
\end{Highlighting}
\end{Shaded}

\begin{verbatim}
## [1] "numeric"
\end{verbatim}

\begin{Shaded}
\begin{Highlighting}[]
\KeywordTok{print}\NormalTok{(}\KeywordTok{typeof}\NormalTok{(iris))}
\end{Highlighting}
\end{Shaded}

\begin{verbatim}
## [1] "list"
\end{verbatim}

\begin{Shaded}
\begin{Highlighting}[]
\KeywordTok{print}\NormalTok{(}\KeywordTok{mode}\NormalTok{(iris))}
\end{Highlighting}
\end{Shaded}

\begin{verbatim}
## [1] "list"
\end{verbatim}

\begin{Shaded}
\begin{Highlighting}[]
\KeywordTok{print}\NormalTok{(}\KeywordTok{class}\NormalTok{(iris))}
\end{Highlighting}
\end{Shaded}

\begin{verbatim}
## [1] "data.frame"
\end{verbatim}

\begin{Shaded}
\begin{Highlighting}[]
\KeywordTok{print}\NormalTok{(}\KeywordTok{typeof}\NormalTok{(sum))}
\end{Highlighting}
\end{Shaded}

\begin{verbatim}
## [1] "builtin"
\end{verbatim}

\begin{Shaded}
\begin{Highlighting}[]
\KeywordTok{print}\NormalTok{(}\KeywordTok{mode}\NormalTok{(sum))}
\end{Highlighting}
\end{Shaded}

\begin{verbatim}
## [1] "function"
\end{verbatim}

\begin{Shaded}
\begin{Highlighting}[]
\KeywordTok{print}\NormalTok{(}\KeywordTok{class}\NormalTok{(sum))}
\end{Highlighting}
\end{Shaded}

\begin{verbatim}
## [1] "function"
\end{verbatim}

Usually, there's no need to fuss about these differences: just use the
\texttt{is.*()} family of functions. Give it a try:

\begin{Shaded}
\begin{Highlighting}[]
\KeywordTok{is.numeric}\NormalTok{(a)}
\end{Highlighting}
\end{Shaded}

\begin{verbatim}
## [1] TRUE
\end{verbatim}

\begin{Shaded}
\begin{Highlighting}[]
\KeywordTok{is.data.frame}\NormalTok{(iris)}
\end{Highlighting}
\end{Shaded}

\begin{verbatim}
## [1] TRUE
\end{verbatim}

We can also coerce objects to take on a different form, typically using
the \texttt{as.*()} family of functions. We can't always coerce! You'll
get a sense of this over time, but try:

\begin{itemize}
\tightlist
\item
  Coercing a number to a character.
\item
  Coercing a character to a number.
\item
  Coercing a number to a data.frame. \texttt{letters} to a data.frame.
\end{itemize}

\begin{Shaded}
\begin{Highlighting}[]
\KeywordTok{as.character}\NormalTok{(}\DecValTok{100}\NormalTok{)}
\end{Highlighting}
\end{Shaded}

\begin{verbatim}
## [1] "100"
\end{verbatim}

\begin{Shaded}
\begin{Highlighting}[]
\KeywordTok{as.numeric}\NormalTok{(}\StringTok{"100"}\NormalTok{)}
\end{Highlighting}
\end{Shaded}

\begin{verbatim}
## [1] 100
\end{verbatim}

\begin{Shaded}
\begin{Highlighting}[]
\KeywordTok{as.data.frame}\NormalTok{(letters)}
\end{Highlighting}
\end{Shaded}

\begin{verbatim}
##    letters
## 1        a
## 2        b
## 3        c
## 4        d
## 5        e
## 6        f
## 7        g
## 8        h
## 9        i
## 10       j
## 11       k
## 12       l
## 13       m
## 14       n
## 15       o
## 16       p
## 17       q
## 18       r
## 19       s
## 20       t
## 21       u
## 22       v
## 23       w
## 24       x
## 25       y
## 26       z
\end{verbatim}

There is also a slight difference between coercion and conversion, but
this is usually not important.

\subsection{Vectors}\label{vectors}

Vectors store multiple entries of a data type. You'll discover that they
show up just about everywhere in R, so they're fundamental and extremely
important.

\subsubsection{Vector Construction and Basic
Subsetting}\label{vector-construction-and-basic-subsetting}

We've seen vectors as columns of data frames:

\begin{Shaded}
\begin{Highlighting}[]
\NormalTok{mtcars}\OperatorTok{$}\NormalTok{hp}
\end{Highlighting}
\end{Shaded}

\begin{verbatim}
##  [1] 110 110  93 110 175 105 245  62  95 123 123 180 180 180 205 215 230
## [18]  66  52  65  97 150 150 245 175  66  91 113 264 175 335 109
\end{verbatim}

Use the \texttt{c()} function to make a vector consisting of the course
code (\texttt{"STAT"} and \texttt{545}). Notice the coercion. Vectors
must be homogeneous.

\begin{Shaded}
\begin{Highlighting}[]
\NormalTok{(course <-}\StringTok{ }\KeywordTok{c}\NormalTok{(}\StringTok{"STAT"}\NormalTok{, }\DecValTok{545}\NormalTok{))}
\end{Highlighting}
\end{Shaded}

\begin{verbatim}
## [1] "STAT" "545"
\end{verbatim}

Subset the first entry. Remove the first entry. Note the base-1 system.

\begin{Shaded}
\begin{Highlighting}[]
\NormalTok{course[}\DecValTok{1}\NormalTok{]}
\end{Highlighting}
\end{Shaded}

\begin{verbatim}
## [1] "STAT"
\end{verbatim}

\begin{Shaded}
\begin{Highlighting}[]
\NormalTok{course[}\OperatorTok{-}\DecValTok{1}\NormalTok{]}
\end{Highlighting}
\end{Shaded}

\begin{verbatim}
## [1] "545"
\end{verbatim}

\begin{Shaded}
\begin{Highlighting}[]
\NormalTok{course}
\end{Highlighting}
\end{Shaded}

\begin{verbatim}
## [1] "STAT" "545"
\end{verbatim}

Use \texttt{\textless{}-} to change the second entry to ``545A''. Using
the same approach, add a third entry, ``S01''.

\begin{Shaded}
\begin{Highlighting}[]
\NormalTok{course[}\DecValTok{2}\NormalTok{] <-}\StringTok{ "545A"}
\NormalTok{course[}\DecValTok{3}\NormalTok{] <-}\StringTok{ "s01"}
\NormalTok{course}
\end{Highlighting}
\end{Shaded}

\begin{verbatim}
## [1] "STAT" "545A" "s01"
\end{verbatim}

Subset the first and third entry. Order matters! Subset the third and
first entry.

\begin{Shaded}
\begin{Highlighting}[]
\NormalTok{course[}\KeywordTok{c}\NormalTok{(}\DecValTok{3}\NormalTok{,}\DecValTok{1}\NormalTok{)]}
\end{Highlighting}
\end{Shaded}

\begin{verbatim}
## [1] "s01"  "STAT"
\end{verbatim}

Explore integer sequences, especially negatives and directions.
Especially \texttt{1:0} that might show up in loops!

\begin{Shaded}
\begin{Highlighting}[]
\DecValTok{3}\OperatorTok{:}\DecValTok{10}
\end{Highlighting}
\end{Shaded}

\begin{verbatim}
## [1]  3  4  5  6  7  8  9 10
\end{verbatim}

\begin{Shaded}
\begin{Highlighting}[]
\DecValTok{10}\OperatorTok{:-}\DecValTok{5}
\end{Highlighting}
\end{Shaded}

\begin{verbatim}
##  [1] 10  9  8  7  6  5  4  3  2  1  0 -1 -2 -3 -4 -5
\end{verbatim}

\begin{Shaded}
\begin{Highlighting}[]
\CommentTok{#Vector of length zero}
\KeywordTok{seq_len}\NormalTok{(}\DecValTok{0}\NormalTok{)}
\end{Highlighting}
\end{Shaded}

\begin{verbatim}
## integer(0)
\end{verbatim}

Singletons are also vectors. Check using \texttt{is.vector}.

\begin{Shaded}
\begin{Highlighting}[]
\KeywordTok{is.vector}\NormalTok{(}\DecValTok{6}\NormalTok{)}
\end{Highlighting}
\end{Shaded}

\begin{verbatim}
## [1] TRUE
\end{verbatim}

\subsubsection{Vectorization and
Recycling}\label{vectorization-and-recycling}

A key aspect of R is its vectorization. Let's work with the vector
following vector:

\begin{Shaded}
\begin{Highlighting}[]
\NormalTok{a <-}\StringTok{ }\DecValTok{7}\OperatorTok{:-}\DecValTok{2}
\NormalTok{a}
\end{Highlighting}
\end{Shaded}

\begin{verbatim}
##  [1]  7  6  5  4  3  2  1  0 -1 -2
\end{verbatim}

\begin{Shaded}
\begin{Highlighting}[]
\NormalTok{(n <-}\StringTok{ }\KeywordTok{length}\NormalTok{(a))}
\end{Highlighting}
\end{Shaded}

\begin{verbatim}
## [1] 10
\end{verbatim}

Square each component:

\begin{Shaded}
\begin{Highlighting}[]
\NormalTok{a}\OperatorTok{^}\DecValTok{2}
\end{Highlighting}
\end{Shaded}

\begin{verbatim}
##  [1] 49 36 25 16  9  4  1  0  1  4
\end{verbatim}

Multiply each component by 1 through its length:

\begin{Shaded}
\begin{Highlighting}[]
\NormalTok{a }\OperatorTok{*}\StringTok{ }\DecValTok{1}\OperatorTok{:}\DecValTok{10}
\end{Highlighting}
\end{Shaded}

\begin{verbatim}
##  [1]   7  12  15  16  15  12   7   0  -9 -20
\end{verbatim}

It's important to know that R will silently recycle! Unless the length
of one vector is not divisible by the other. Let's see:

\begin{Shaded}
\begin{Highlighting}[]
\NormalTok{a }\OperatorTok{*}\StringTok{ }\DecValTok{1}\OperatorTok{:}\DecValTok{3} \CommentTok{#duplicating 1:3 over and over because it's not long enough}
\end{Highlighting}
\end{Shaded}

\begin{verbatim}
## Warning in a * 1:3: longer object length is not a multiple of shorter
## object length
\end{verbatim}

\begin{verbatim}
##  [1]  7 12 15  4  6  6  1  0 -3 -2
\end{verbatim}

\begin{Shaded}
\begin{Highlighting}[]
\NormalTok{a }\OperatorTok{*}\StringTok{ }\DecValTok{1}\OperatorTok{:}\DecValTok{2}
\end{Highlighting}
\end{Shaded}

\begin{verbatim}
##  [1]  7 12  5  8  3  4  1  0 -1 -4
\end{verbatim}

This is true of comparison operators, too. Make a vector of logicals
using a comparison operator.

Now try a boolean operator. Note that \&\& and \textbar{}\textbar{} are
NOT vectorized!

Recycling works with assignment, too. Replace the entire vector a with
1:2 repeated:

\subsubsection{Special Subsetting}\label{special-subsetting}

We can subset vectors by names and logicals, too.

Recall the course vector:

\begin{Shaded}
\begin{Highlighting}[]
\NormalTok{course <-}\StringTok{ }\KeywordTok{c}\NormalTok{(}\StringTok{"STAT"}\NormalTok{, }\StringTok{"545A"}\NormalTok{, }\StringTok{"S01"}\NormalTok{)}
\end{Highlighting}
\end{Shaded}

Let's give the components some names (``subject'', ``code'', and
``section'') using three methods:

\begin{enumerate}
\def\labelenumi{\arabic{enumi}.}
\tightlist
\item
  Using the setNames function.
\end{enumerate}

\begin{itemize}
\tightlist
\item
  Notice that the vector does not change!!
\end{itemize}

\begin{enumerate}
\def\labelenumi{\arabic{enumi}.}
\setcounter{enumi}{1}
\item
  Using the names function with \texttt{\textless{}-}. Also, just
  explore the names function.
\item
  Re-constructing the vector, specifying names within \texttt{c()}.
\end{enumerate}

Subset the entry labelled ``section'' and ``subject''.

Amazingly, we can also subset by a vector of logicals (which will be
recycled!). Let's work with our integer sequence vector again:

\begin{Shaded}
\begin{Highlighting}[]
\NormalTok{(a <-}\StringTok{ }\DecValTok{7}\OperatorTok{:-}\DecValTok{2}\NormalTok{)}
\end{Highlighting}
\end{Shaded}

\begin{verbatim}
##  [1]  7  6  5  4  3  2  1  0 -1 -2
\end{verbatim}

\begin{Shaded}
\begin{Highlighting}[]
\NormalTok{(n <-}\StringTok{ }\KeywordTok{length}\NormalTok{(a))}
\end{Highlighting}
\end{Shaded}

\begin{verbatim}
## [1] 10
\end{verbatim}

\subsection{Lists}\label{lists}

Unlike vectors, which are atomic/homogeneous, a list in R is
heterogeneous.

Try storing the course code (\texttt{"STAT"} and \texttt{545}) again,
but this time in a list. Use the \texttt{list()} function.

Lists can hold pretty much anything, and can also be named. Let's use
the following list:

\begin{Shaded}
\begin{Highlighting}[]
\NormalTok{(my_list <-}\StringTok{ }\KeywordTok{list}\NormalTok{(}\DataTypeTok{year=}\DecValTok{2018}\NormalTok{, }\DataTypeTok{instructor=}\KeywordTok{c}\NormalTok{(}\StringTok{"Vincenzo"}\NormalTok{, }\StringTok{"Coia"}\NormalTok{), }\DataTypeTok{fav_fun=}\NormalTok{typeof))}
\end{Highlighting}
\end{Shaded}

\begin{verbatim}
## $year
## [1] 2018
## 
## $instructor
## [1] "Vincenzo" "Coia"    
## 
## $fav_fun
## function (x) 
## .Internal(typeof(x))
## <bytecode: 0x7fbceb32cdf0>
## <environment: namespace:base>
\end{verbatim}

Subsetting a list works similarly to vectors. Try subsetting the first
element of \texttt{my\_list}; try subsettig the first \emph{component}
of the list. Notice the difference!

Try also subsetting by name:

Smells a little like \texttt{data.frame}s. It turns out a
\texttt{data.frame} is a special type of list:

\begin{Shaded}
\begin{Highlighting}[]
\NormalTok{(small_df <-}\StringTok{ }\NormalTok{tibble}\OperatorTok{::}\KeywordTok{tibble}\NormalTok{(}\DataTypeTok{x=}\DecValTok{1}\OperatorTok{:}\DecValTok{5}\NormalTok{, }\DataTypeTok{y=}\NormalTok{letters[}\DecValTok{1}\OperatorTok{:}\DecValTok{5}\NormalTok{]))}
\end{Highlighting}
\end{Shaded}

\begin{verbatim}
## # A tibble: 5 x 2
##       x y    
##   <int> <chr>
## 1     1 a    
## 2     2 b    
## 3     3 c    
## 4     4 d    
## 5     5 e
\end{verbatim}

\begin{Shaded}
\begin{Highlighting}[]
\KeywordTok{is.list}\NormalTok{(small_df)}
\end{Highlighting}
\end{Shaded}

\begin{verbatim}
## [1] TRUE
\end{verbatim}

\begin{Shaded}
\begin{Highlighting}[]
\KeywordTok{as.list}\NormalTok{(small_df)}
\end{Highlighting}
\end{Shaded}

\begin{verbatim}
## $x
## [1] 1 2 3 4 5
## 
## $y
## [1] "a" "b" "c" "d" "e"
\end{verbatim}

Note that there's a difference between a list of one object, and that
object itself! This is different from vectors.

\begin{Shaded}
\begin{Highlighting}[]
\KeywordTok{identical}\NormalTok{(}\KeywordTok{list}\NormalTok{(}\DecValTok{4}\NormalTok{), }\DecValTok{4}\NormalTok{)}
\end{Highlighting}
\end{Shaded}

\begin{verbatim}
## [1] FALSE
\end{verbatim}

\begin{Shaded}
\begin{Highlighting}[]
\KeywordTok{identical}\NormalTok{(}\KeywordTok{c}\NormalTok{(}\DecValTok{4}\NormalTok{), }\DecValTok{4}\NormalTok{)}
\end{Highlighting}
\end{Shaded}

\begin{verbatim}
## [1] TRUE
\end{verbatim}

\section{Part II}\label{part-ii}

\subsection{Global Environment}\label{global-environment}

When you assign variables in R, the variable name and contents are
stored in an R environment called a global environment.

See what's in the Global Environment by:

\begin{itemize}
\tightlist
\item
  Executing \texttt{ls()}.
\item
  Looking in RStudio, in the ``Environments'' pane.
\end{itemize}

Making an assignment ``binds'' an object to a name within an
environment. For example, writing \texttt{a\ \textless{}-\ 5} assigns
the object \texttt{5} to the name \texttt{a} in the global environment.

The act of ``searching for the right object to return'' is called
scoping.

By the way: the global environment is an object, too! It's the output of
\texttt{globalenv()}, and is also stored in the variable
\texttt{.GlobalEnv}:

\begin{Shaded}
\begin{Highlighting}[]
\KeywordTok{globalenv}\NormalTok{()}
\end{Highlighting}
\end{Shaded}

\begin{verbatim}
## <environment: R_GlobalEnv>
\end{verbatim}

\begin{Shaded}
\begin{Highlighting}[]
\NormalTok{.GlobalEnv}
\end{Highlighting}
\end{Shaded}

\begin{verbatim}
## <environment: R_GlobalEnv>
\end{verbatim}

\subsection{The Search Path}\label{the-search-path}

How does R know what \texttt{iris} is, yet \texttt{iris} does not appear
in the global environment? What about functions like \texttt{length},
\texttt{sum}, and \texttt{print} (which are all objects, too)?

Let's explore.

\begin{enumerate}
\def\labelenumi{\arabic{enumi}.}
\item
  Each package has its own environment.

  \begin{itemize}
  \tightlist
  \item
    Install and load the \texttt{pryr} package, and use \texttt{ls()} to
    list its bindings (its name is ``package:pryr'').
  \end{itemize}
\item
  There's a difference between an \emph{environment} and its
  \emph{name}. Get the environment with name ``package:pryr'' using the
  \texttt{as.environment()} function.
\item
  Each environment has a parent. Use \texttt{parent.env()} to find the
  parent of the global environment.
\item
  There are packages that come pre-loaded with R, and they're loaded in
  a sequence called the search path. Use \texttt{search()} to identify
  that path; then see it in RStudio.
\end{enumerate}

First scoping rule: R looks to the parent environment if it can't find
an object where it is.

\begin{enumerate}
\def\labelenumi{\arabic{enumi}.}
\setcounter{enumi}{4}
\item
  Use \texttt{pryr::where()} to determine where the first binding to the
  name \texttt{iris} is located.
\item
  Override \texttt{iris} with, say, a numeric. Now \texttt{where()} is
  it? Can you extract the original?
\item
  Override \texttt{sum} with, say, a numeric. \texttt{where()} is
  \texttt{sum} now? Can you still use the original \texttt{sum()}
  function?
\end{enumerate}

Special scoping rule for functions! R knows whether or not to look for a
function.

\begin{enumerate}
\def\labelenumi{\arabic{enumi}.}
\setcounter{enumi}{7}
\tightlist
\item
  Look at the source code of the \texttt{pryr:where()} function. It
  contains a line that creates a binding for ``env''. Why, then, is
  \texttt{env} nowhere to be found? Answer: execution environments.
\end{enumerate}


\end{document}
